\begin{figure}
\centering
	% To include a figure from a file named example.*
	% Allowable file formats are eps or ps if compiling using latex
	% or pdf, png, jpg if compiling using pdflatex
	\includegraphics[width=0.5\textwidth]{tau_L_correlation_type.png}
    \caption{$\tau$-L correlation for CLAGNs and quasars, where $\tau$ is time lag between V band and $W$1 band and L is bolometric luminosity. Grey circles represent PG quasars \citep{2019ApJ...886...33L}. The grey dashed line repserent the best fitting of $\tau$-L correlation for PG quasars with a slope of $\sim$ 0.47 in \citet{2019ApJ...886...33L}.
    Stars represent Type 1 AGN \citep[][]{2014ApJ...788..159K,2019ApJ...886...33L}. The time lags for Type 1s are scaled from $\tau_\mathrm{K}$ to $\tau_\mathrm{W1}$ by multiplying a factor of 5/3 according to \citet[][]{2019ApJ...886...33L}. Square represents the type 2 AGN NGC 2110 for X-ray and MIR time lag \citep[see ][]{2020MNRAS.495.2921N}. The black straight line represents the best-fitting for Type 1s, Type 2s, and CLAGNs, with a slope of $\sim$ 0.36. The red dashed lines represent the scatter of $\sim$0.17 dex.  }
    \label{fig:tau_L}. 
\end{figure}


\begin{table}
 \caption{Dust reverberation time lag of CLAGNs.
}
 \label{table_lag}
 \begin{center}
 \begin{tabular}{lccccc}
 \hline\hline
Name & log($L_{bol}/L_{\odot}$) & $\tau$~(days) &Type  & references\\ \hline 
NGC 1566 & 9.91 & $37.1^{+25.7}_{-11.9}$ & O & This work \\
Mrk 6 & 11.01 & $142.1^{+30.0}_{-11.5}$ & O & This work \\
Mrk 926 & 12.08 & $213.8^{+25.3}_{-31.0}$ & O & This work \\
PG 1535+547 & 11.56 & $153.2^{+21.2}_{-25.3}$ & X & This work \\
NGC 2110 & 10.95 & $153.2^{+21.2}_{-25.3}$ & Type 2 & \citet{2020MNRAS.495.2921N} \\
Mrk 335 & 11.03 & $ 167.1 \pm 5.60$ & Type 1 & \citet{2014ApJ...788..159K} \\
Mrk 590 & 10.25 & $ 33.8 \pm 4.20$ & O & \citet{2014ApJ...788..159K} \\
IRAS 03450+0055 & 11.34 & $ 158.3 \pm 5.40$ & Type 1 & \citet{2014ApJ...788..159K} \\
Akn 120 & 11.67 & $ 140.9 \pm 17.20$ & Type 1 & \citet{2014ApJ...788..159K} \\
MCG +08-11-011 & 10.88 & $ 73.5 \pm 1.40$ & Type 1 & \citet{2014ApJ...788..159K} \\
Mrk 79 & 10.77 & $ 65.6 \pm 5.00$ & Type 1 & \citet{2014ApJ...788..159K} \\
Mrk 110 & 11.03 & $ 119.7 \pm 5.50$ & Type 1 & \citet{2014ApJ...788..159K} \\
NGC 3227 & 9.60 & $ 14.6 \pm 0.70$ & Type 1 & \citet{2014ApJ...788..159K} \\
NGC 3516 & 10.03 & $ 73.1 \pm 4.00$ & O & \citet{2014ApJ...788..159K} \\
Mrk 744 & 9.23 & $ 20.0 \pm 2.20$ & Type 1 & \citet{2014ApJ...788..159K} \\
NGC 4051 & 9.08 & $ 16.5 \pm 0.60$ & X & \citet{2014ApJ...788..159K} \\
NGC 4151 & 10.03 & $ 48.3 \pm 0.50$ & O & \citet{2014ApJ...788..159K} \\
NGC 4593 & 9.95 & $ 41.6 \pm 0.90$ & Type 1 & \citet{2014ApJ...788..159K} \\
NGC 5548 & 10.29 & $ 60.9 \pm 0.30$ & O & \citet{2014ApJ...788..159K} \\
Mrk 817 & 11.12 & $ 93.0 \pm 8.90$ & Type 1 & \citet{2014ApJ...788..159K} \\
Mrk 509 & 11.63 & $ 120.7 \pm 1.80$ & Type 1 & \citet{2014ApJ...788..159K} \\
NGC 7469 & 10.69 & $ 88.0 \pm 0.60$ & X & \citet{2014ApJ...788..159K} \\
Mrk 335 & 11.17 & 137.7 & Type 1 & \citet{2019ApJ...886...33L} \\
IRAS 03450+0055 & 11.46 & 129.1 & Type 1 & \citet{2019ApJ...886...33L} \\
MCG +08-11-011 & 11.76 & 89.5 & Type 1 & \citet{2019ApJ...886...33L} \\
Mrk 79 & 10.93 & 74.6 & Type 1 & \citet{2019ApJ...886...33L} \\
Mrk 110 & 11.17 & 63.7 & Type 1 & \citet{2019ApJ...886...33L} \\
NGC 3227 & 9.85 & 7.0 & Type 1 & \citet{2019ApJ...886...33L} \\
NGC 3516 & 10.26 & 53.6 & O & \citet{2019ApJ...886...33L} \\
Mrk 744 & 9.48 & 26.6 & Type 1 & \citet{2019ApJ...886...33L} \\
NGC 4051 & 9.30 & 12.9 & X & \citet{2019ApJ...886...33L} \\
NGC 4151 & 10.26 & 52.2 & O & \citet{2019ApJ...886...33L} \\
NGC 4593 & 10.17 & 62.1 & Type 1 & \citet{2019ApJ...886...33L} \\
NGC 5548 & 10.49 & 50.5 & O & \citet{2019ApJ...886...33L} \\
Mrk 817 & 11.26 & 91.6 & Type 1 & \citet{2019ApJ...886...33L} \\
Mrk 509 & 11.71 & 122.2 & Type 1 & \citet{2019ApJ...886...33L} \\
NGC 7469 & 10.87 & 56.1 & X & \citet{2019ApJ...886...33L} \\
\hline\hline
\end{tabular}
\end{center}
Note. The table lists source name, bolometric luminosity, dust reverberation time lag, CLAGN types (``O'' for optical selected CLAGN and ``X'' for X-ray selected CLAGN) and references. The time lag for type 2 AGN NGC 2110 is between X-ray (2-20 keV) band and $W$1 band. The time lag for NGC 1566 is between V band and $W$1 band in this work. The time lags results from \citet{2014ApJ...788..159K} and \citet{2019ApJ...886...33L} are between V band and K band. 
\end{table}

